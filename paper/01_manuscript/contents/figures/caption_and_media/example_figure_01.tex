%% -*- coding: utf-8 -*-
%% Figure caption template
%% Figure ID: 01
%%
%% GUIDELINES:
%% - Describe what the figure shows and its key message
%% - Explain each panel (A, B, C, etc.) systematically
%% - Define all symbols, colors, line styles, error bars
%% - Keep technical details in caption, interpretation in main text
%% - Reference in text as Figure~\ref{fig:example01}

%% SUPPORTED FORMATS:
%% - Images: .jpg, .png, .pdf, .tif
%% - Mermaid diagrams: .mmd (auto-converted)
%% - Place files in this directory

%% EXAMPLE FIGURE:
\begin{figure}[htbp]
\centering
\includegraphics[width=0.85\textwidth]{figures/caption_and_media/example_figure_01.png}
\caption{\textbf{Example experimental workflow and results.}
(\textbf{A}) Schematic diagram showing the experimental design with sample sizes
(n=30 per group) and timeline.
(\textbf{B}) Representative data showing measurement over time. Blue line: control
group; red line: treatment group. Shaded areas represent 95\% confidence intervals.
(\textbf{C}) Quantitative comparison between groups. Error bars indicate standard
deviation. **P < 0.01, two-tailed t-test.
(\textbf{D}) Correlation analysis between variables X and Y (Pearson's r = 0.85,
P < 0.001, n=60).}
\label{fig:example01}
\end{figure}

%% TIPS:
%% - Use \textbf{} for panel labels (A), (B), (C)
%% - Define statistical tests and significance levels
%% - Specify sample sizes where relevant
%% - Explain what error bars/shading represent
%% - Use consistent terminology with main text

%%%% EOF
