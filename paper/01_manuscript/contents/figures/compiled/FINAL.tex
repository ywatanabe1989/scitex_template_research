% Generated by compile_figure_tex_files()
% This file includes all figure files in order

% Figure 0
\begin{figure*}[h!]
    \pdfbookmark[2]{Figure 0}{.0}
    \centering
    \includegraphics[width=0.95\textwidth]{./01_manuscript/contents/figures/caption_and_media/jpg_for_compilation/00_experimental_design.jpg}
    \caption{\textbf{
FIGURE TITLE HERE
}
\smallskip
\
FIGURE LEGEND HERE.
}
    \label{fig:0_experimental_design}
\end{figure*}

% Figure 1
\begin{figure*}[htbp]
    \pdfbookmark[2]{Figure 1}{.1}
    \centering
    \includegraphics[width=0.95\textwidth]{./01_manuscript/contents/figures/caption_and_media/jpg_for_compilation/01_demographic_data.jpg}
    \caption{\textbf{Seizure Demographics Overview}\\
\smallskip
\textbf{A.} Seizure raster plot demonstrating temporal patterns and frequency variations.
\textbf{B.} Interictal control periods randomly sampled with matching number of events, duration, time of day
\textbf{C.} Hourly seizure distribution showing circadian patterns across patient cohort.}
    \label{fig:1_demographic_data}
\end{figure*}

% Figure 2
\begin{figure*}[htbp]
    \pdfbookmark[2]{Figure 2}{.2}
    \centering
    \includegraphics[width=0.95\textwidth]{./01_manuscript/contents/figures/caption_and_media/jpg_for_compilation/02_pac_basic.jpg}
    \caption{\textbf{Time-dependent, pre-ictal PAC features}\\
\smallskip
\textbf{A.} Z-scores of PAC comodulograms against surrogate data over time, triggered to seizure onsets.
\textbf{B.} Seventeen descriptive metrics of PAC calculations; shaded lines show median ± IQR and raw values as lines.
\textbf{C.} Effect sizes of Brunner-Munzel tests comparing Ashman's D (bimodality metric) between seizure and control groups across time windows. Red dotted lines show linear fits to effect sizes in the time window [-1024, 0) minutes from seizure onset, with R² values indicating goodness of fit.
}
    \label{fig:2_pac_basic}
\end{figure*}

% Figure 3
\begin{figure*}[htbp]
    \pdfbookmark[2]{Figure 3}{.3}
    \centering
    \includegraphics[width=0.95\textwidth]{./01_manuscript/contents/figures/caption_and_media/jpg_for_compilation/03_classification_as_prediction.jpg}
    \caption{\textbf{Classification as Prediction}\\
\smallskip
\textbf{A.} Data splitting for pseudo-prospective prediction design.
\textbf{B.} Feature selection using validation dataset.
\textbf{C.} Confusion matrix.
\textbf{D.} Classification accuracy.
\textbf{E.} Prediction metrics.
}
    \label{fig:3_classification_as_prediction}
\end{figure*}

% Figure 4
\begin{figure*}[htbp]
    \pdfbookmark[2]{Figure 4}{.4}
    \centering
    \includegraphics[width=0.95\textwidth]{./01_manuscript/contents/figures/caption_and_media/jpg_for_compilation/04_feature_importance.jpg}
    \caption{\textbf{Feature Importance}\\
\smallskip
FIGURE LEGEND HERE.
}
    \label{fig:4_feature_importance}
\end{figure*}

