%% -*- coding: utf-8 -*-
%% Results template

\section{Results}

%% ============================================================
%% RESULTS STRUCTURE GUIDE:
%% ============================================================
%% Present findings objectively without interpretation
%% Organize in a logical flow that tells a story
%% Use subsections to group related findings
%% Include specific numbers, statistics, and effect sizes
%% Reference figures and tables appropriately
%% Typical structure:
%%   - Dataset characteristics
%%   - Descriptive/exploratory results
%%   - Main findings
%%   - Secondary analyses
%%   - Validation results
%% ============================================================

\subsection{Dataset Characteristics}
% Provide overview of your dataset and sample characteristics
% Include descriptive statistics, demographics, or system properties
% Reference relevant tables and figures

% Example:
% The dataset comprised [description] (Figure~\ref{fig:dataset_overview}).
% [Characteristic 1] ranged from [min] to [max] (mean: [value]±[SD]).
% Analysis included [N] samples/participants/measurements distributed across
% [categories] (Table~\ref{tab:characteristics}).

[Describe your dataset characteristics here with specific numbers and statistics.
Reference Figure 1 and Table 1.]


\subsection{[First Main Result Category]}
% Present your primary findings
% Use descriptive subsection titles that indicate the result

% Example structure:
% [Analysis method] revealed [finding] (Figure~\ref{fig:main_result}). Specifically,
% [metric 1] showed [description] (value: [X]±[Y], p < [Z]). [Metric 2] demonstrated
% [description] across [conditions] (Table~\ref{tab:detailed_results}).

[Present your first set of main results here with specific quantitative findings.
Include references to figures and tables.]


\subsection{[Second Main Result Category]}
% Continue with additional key findings
% Maintain focus on objective presentation

[Present your second set of main results here.]


\subsection{[Performance/Validation Results]}
% If applicable, present model performance, validation results, or benchmarking

% Example for ML studies:
% [Model type] achieved [performance metric 1] of [value]±[SD] across [N] folds
% (Figure~\ref{fig:performance}). [Metric 2] ranged from [min] to [max] (mean: [value]±[SD]),
% with [N] of [total] cases exceeding the [threshold] threshold. Performance varied
% [description] across [conditions] (coefficient of variation: [X]%).

[Present performance or validation results here if applicable.]


\subsection{[Additional Analyses]}
% Present any secondary or supporting analyses
% Feature importance, sensitivity analyses, subgroup analyses, etc.

[Present additional analyses here if applicable.]


% ==================================================
% FIGURE AND TABLE PLACEHOLDERS (Update as needed)
% ==================================================
% Below are placeholder references - update these to match your actual figures and tables
%
% Figure 1: Dataset overview and characteristics
% Figure 2: Main experimental results
% Figure 3: Detailed analysis and patterns
% Figure 4: Performance metrics and validation
% Figure 5: Additional analyses
%
% Table 1: Dataset characteristics and demographics
% Table 2: Statistical test results
% Table 3: Performance metrics summary
% Table 4: Additional numerical results

\label{sec:results}

%%%% EOF
