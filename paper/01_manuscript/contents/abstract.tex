%% -*- coding: utf-8 -*-
%% Abstract template
%% The abstract should be 150-300 words (check your target journal's requirements)
\begin{abstract}
  \pdfbookmark[1]{Abstract}{abstract}

%% ============================================================
%% ABSTRACT STRUCTURE GUIDE (Based on Nature-style format):
%% ============================================================
%% An effective abstract provides a clear, concise, and accessible summary
%% to a broad audience of scientists. Follow this 7-part structure:
%%
%% [1. Basic Introduction] (1-2 sentences)
%%     Provide a basic introduction to the field, comprehensible to
%%     scientists in any discipline.
%%
%% [2. Detailed Background] (2-3 sentences)
%%     Give more detailed background, comprehensible to scientists
%%     in related disciplines.
%%
%% [3. General Problem] (1 sentence)
%%     Clearly state the general problem being addressed by this
%%     particular study.
%%
%% [4. Main Result] (1 sentence)
%%     Summarize the main result using "here we show" or equivalent.
%%
%% [5. Results with Comparisons] (2-3 sentences)
%%     Explain what the main result reveals in direct comparison to
%%     what was thought to be the case previously, or how it adds
%%     to previous knowledge. Include specific quantitative results.
%%
%% [6. General Context] (1-2 sentences)
%%     Place your results in a broader context, linking findings
%%     to the wider field of study.
%%
%% [7. Broader Perspective] (2-3 sentences, optional)
%%     Provide broader perspective, comprehensible to scientists
%%     in any discipline. Discuss implications and future impact.
%% ============================================================

%% WRITING GUIDELINES:
%% - Write as coherent, cohesive sentences without line breaks
%% - Use present tense for general facts (supported by multiple works)
%% - Use past tense for specific prior research
%% - Use past tense for results/observations in this study
%% - Maintain quantitative measurements precisely as they are
%% - Explicitly indicate species with sample sizes where relevant

%% EXAMPLE STRUCTURE:
%% [1] [Research area] represents a significant challenge in [domain], affecting
%% approximately [scale/number] people/systems worldwide. [2] While previous studies
%% have demonstrated [existing knowledge], comprehensive characterization remains
%% limited by [specific limitations]. Current approaches achieve [metrics] but face
%% challenges including [limitation 1], [limitation 2], and [limitation 3]. [3] The
%% inability to [specific problem] has hindered [desired outcome]. [4] Here we show
%% that [your main finding]. [5] We [method 1] and [method 2], extracting [measurements]
%% at [scales]. We identified [specific finding 1] and [specific finding 2], achieving
%% [performance metric 1] of [X±Y]% and [performance metric 2] of [Z±W]. [6] These
%% findings establish [your contribution] as a [qualities] approach for [application].
%% [7] Our results provide a foundation for [future directions], potentially
%% transforming [field] from [current state] to [improved state].

[Write your abstract here following the 7-part structure above. Replace this entire
bracketed section with your actual abstract text. Write as a continuous paragraph
without section markers. Keep it concise (150-300 words) and focused on the most
important aspects of your research. Use specific numbers and metrics to make your
findings concrete.]

\end{abstract}

%%%% EOF
