%% -*- coding: utf-8 -*-
%% Introduction template
%% Typical length: 4-8 paragraphs (1000-2000 words)

\section{Introduction}

%% ============================================================
%% INTRODUCTION STRUCTURE GUIDE:
%% ============================================================
%% Paragraph 1: Problem Statement and Significance
%%   - Introduce the broad research area
%%   - Establish the importance and impact of the problem
%%   - Provide context with relevant statistics or facts
%%   - Cite foundational references
%%
%% Paragraph 2: Current State of Knowledge
%%   - Summarize what is currently known
%%   - Discuss existing approaches and their achievements
%%   - Cite relevant literature systematically
%%
%% Paragraph 3: Gaps and Limitations
%%   - Identify specific limitations of current approaches
%%   - Explain why these limitations matter
%%   - Discuss challenges that hinder progress
%%   - Support with citations
%%
%% Paragraph 4: Your Approach (Introduction)
%%   - Introduce your method/biomarker/technique
%%   - Explain its unique advantages
%%   - Describe its theoretical or practical significance
%%   - Cite relevant foundational work
%%
%% Paragraph 5: Study Objectives and Hypotheses
%%   - Clearly state your research hypotheses
%%   - Outline your specific objectives
%%   - Describe your experimental approach at a high level
%%
%% Paragraph 6: Key Findings Preview (optional)
%%   - Briefly preview your main results
%%   - Highlight the significance and implications
%%   - Transition to the Methods section
%% ============================================================

% PARAGRAPH 1: Problem Statement
% Template:
% [Research area] affects approximately [number] people/systems worldwide, representing
% [significance]. [Specific problem], occurring in approximately [percentage], remains
% particularly challenging as [explanation of difficulty]. The development of [solution type]
% represents a critical frontier in [field], offering the potential to transform
% [application area] from [current state] to [improved state].

[Write your problem statement here. Introduce the broad research area, establish its
importance, and provide context with relevant statistics. Replace this bracketed text
with your actual introduction.]


% PARAGRAPH 2: Current Knowledge
% Template:
% Recent advances in [specific area] have demonstrated [achievements], achieving
% [performance metrics] across [scope]. However, existing approaches face several
% limitations including [limitation 1], [limitation 2], and [limitation 3]. These
% limitations underscore the need for [what is needed].

[Summarize the current state of knowledge in your field. Discuss existing approaches
and their achievements. Cite relevant literature.]


% PARAGRAPH 3: Knowledge Gap
% Template:
% Despite [existing progress], [specific gap] remains largely unexplored. Current methods
% are limited by [specific limitation], making [desired outcome] impractical. Long-term
% [characteristic] over [timescale] monitoring periods remains unclear, limiting [application].

[Identify specific gaps and limitations in current approaches. Explain why these
limitations matter for your field.]


% PARAGRAPH 4: Your Approach
% Template:
% [Your method/biomarker] represents a promising approach for [objective], offering
% [specific advantages]. Previous studies have demonstrated [relevant findings], but
% comprehensive analysis has been limited by [constraint]. [Your innovation] addresses
% these challenges by [how it helps].

[Introduce your specific approach, method, or technique. Explain its unique advantages
and theoretical significance.]


% PARAGRAPH 5: Objectives and Hypotheses
% Template:
% In this study, we hypothesized that (i) [hypothesis 1], and (ii) [hypothesis 2].
% To test these hypotheses, we [developed/applied] [your system/method] to [dataset/system],
% comprising [description]. We performed [analysis 1], [analysis 2], and [analysis 3].

[Clearly state your research hypotheses and specific objectives. Outline your
experimental approach at a high level.]


% PARAGRAPH 6: Preview of Findings (optional)
% Template:
% Our findings demonstrate that [main result 1], achieving [metric]. We identified
% [main result 2] that [description], providing [contribution] for [application area].

[Optionally, briefly preview your main results and their significance.]


\label{sec:introduction}

%%%% EOF
